\documentclass[]{article}

\usepackage[margin=1cm,bottom=1cm]{geometry}
\usepackage{hyperref}
\usepackage{enumitem}

\setlength{\parindent}{0pt}
\setlist[itemize]{noitemsep,topsep=0pt}

\newcommand{\br}{\vspace{10pt}}
\newcommand{\brs}{\vspace{3pt}}
\newcommand{\hr}{\vspace{4pt}\hrule\vspace{4pt}}

\begin{document}

\thispagestyle{empty}

{\LARGE JONATHAN LAM}

Software Engineer
%% TODO: add more to this summary line

\href{mailto:jlam55555@gmail.com}{\textbf{jlam55555}@gmail.com} --
\href{https://lambdalambda.ninja}{\textbf{lambdalambda.ninja}} --
\href{https://github.com/jlam55555}{github.com/\textbf{jlam55555}} --
\href{https://linkedin.com/in/jonlamdev}{linkedin.com/in/\textbf{jlam55555}} --
New York, NY

\br
\textbf{EDUCATION}
\hr
\textbf{The Cooper Union for the Advancement of Science and Art}
New York, NY
\hfill
September 2018 - May 2022\\
M.Eng., B.Eng., in Electrical Engineering, Computer Engineering Track;\\
Minor in Computer Science; Cumulative GPA: 3.99/4.00

\brs

\textbf{Coursework} Operating Systems, Compilers, Program Analysis, Computer
Architecture,\\
Cloud Computing, Cybersecurity, Databases, Communications Networks, Deep Learning, AI

\brs

\textbf{Activities \& Awards} Math and CS Tutor, Ping Pong Club President, CUCC
Student Operator, \\ ACM ICPC Participant, IEEE$\times$ACM Club Officer, Norman
Perry Award, Jesse Sherman Award, \\ Howard Flagg Memorial Prize, Harold S. Goldberg
Leadership Prize, Henri D. Dickenson Award

\br

\textbf{EXPERIENCE}
\hr

\textbf{Google Silicon, Pixel TPU Runtime Team}
Mountain View, CA
\hfill
August 2022 - January 2023
\begin{itemize}
\item Developed and presented a prototype C++ ``performance HAL'' that significantly reduces \\
  tail latencies and slightly improves across-the-board latencies when compared to existing runtime.
\item Developed an generalized Perfetto trace analysis framework to improve performance insights.
\item Investigated additional opportunities for performance improvements using standard Linux APIs.
\end{itemize}

\brs

\textbf{University of Michigan, Future of Programming Lab}
Ann Arbor, MI
\hfill
October 2021 - May 2022
\begin{itemize}
\item Designed and implemented (mostly performance-related) improvements to evaluation and hole \\
  instance numbering in Hazel, an experimental live programming environment with typed holes.
\item Demonstrated an exponential speedup in certain examples due to memoization of environments.
\item Prototyped the fill-and-resume optimization initially described in Omar et al. (2019).
\end{itemize}

\brs

\textbf{MathWorks, Software Engineer Intern}
Natick, MA
\hfill
May 2021 - August 2021
\begin{itemize}
\item Collaborated with two teams to develop a R\&D prototype for a new workflow that bridges\\
  existing user-facing interactive editing workflows.
\end{itemize}

\brs

\textbf{The Cooper Union, MATLAB Instructor}
New York, NY
\hfill
February 2021 - May 2021
\begin{itemize}
\item Taught and developed materials for ECE210: MATLAB Seminar, an introduction to\\
  MATLAB with applications from the corequisite course ECE211: Signals and Systems.
\end{itemize}

\brs

\textbf{Express Scripts, Software Engineer Intern}
Bloomfield, CT
\hfill
May 2020 - August 2020
\begin{itemize}
\item Won second-place intern project for a browser extension that encourages better WFH productivity.
\item Refactored redundancies in existing Angular projects to improve deployment speed and consistency.
\end{itemize}

\br

\textbf{PROJECT WORK}
\hr

\textbf{Compiler for an untyped lazy pure functional language}
\hfill
January 2022 - May 2022
\begin{itemize}
\item Implemented a front-end (lexer and LL(1) parser) and two back-ends (template instantiation evaluator \\
  and G-Machine compiler) in Haskell for a compiler for Core, an untyped Haskell-like toy language.
\end{itemize}

\brs

\textbf{C99 Compiler}
\hfill
February 2021 - May 2021
\begin{itemize}
\item Developed a C compiler comprising a lexer (Flex), LALR(1) parser (Bison),
  three-address \\ quad generation, and x86\_64 target code emission
  implementing most of the C99 standard.
\end{itemize}

\brs

\textbf{Variations on a Scheme: Multiple First-Class Continuations}
\hfill
August 2021
\begin{itemize}
\item Explored first-class continuation implementations and use cases, and compared to similar constructs.
\item Extended the call/cc interface in Scheme to support multiple simultaneous continuations using CPS.
\end{itemize}

\brs

\textbf{Intrinsic Dimensions of Objective Landscapes}
\hfill
November 2020 - December 2020
\begin{itemize}
\item Extended the work of Li et al. (2018) to find a lower minimum parameterization of the \\
  objective landscape of deep neural networks.
\item Achieved a lower parameterization by using a nonlinear Fourier-coefficients transform.
\end{itemize}

\brs

\textbf{VEIKK Digitizer Driver}
\hfill
July 2019 - August 2020
\begin{itemize}
\item Developed an open-source Linux driver for VEIKK digitizer tablets using USBHID kernel
  API.
\item Built an accompanying C++ configuration GUI tool featuring button, pressure, and screen \\
  mappings; employs systemd libevdev, udev, uinput, Qt5, and (q)dbus Linux APIs.
\end{itemize}

\br

\textbf{TECHNICAL SKILLS}
\hr

\textbf{Languages} C++, C, Python, Javascript, Java, Scheme, SQL, Haskell, OCaml,
MATLAB, x86\_64 Assembly, Rust, Golang

\brs

\textbf{Familiar Technologies} Linux, BSD, Node.js, TypeScript, Angular 2+, Vue, React, Sass, Redux/NgRx, jQuery, Matplotlib, Numpy, Pandas, Tensorflow, KVM/QEMU, \LaTeX, CUDA, AWS, MEAN/MERN/LAMP, LLVM, Perfetto, (simple)perf

\end{document}

%%% Local Variables:
%%% mode: latex
%%% TeX-master: t
%%% End:
